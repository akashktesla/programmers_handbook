\documentclass[12pt]{extarticle}
\usepackage{lmodern} % Required for inserting images
\usepackage{graphicx} % Required for inserting images
\usepackage{amsmath}
\usepackage{amssymb}
\usepackage{amsfonts}
\usepackage{float}

\title{Programmer's HandBook}
\author{Akash Tesla}
\date{July 2025}

\begin{document}
\tableofcontents
\newpage
\maketitle

\section*{Graph Theory}
\section{Types of Graphs}
\subsection{Undirected Graph}
A Undirected graph is a graph in which edges have no orientation. the edge (u,v) is identical
to (v,u)

\subsection{Directed Graph(Digraph)}
A directed graph is a graph in which edges have orientation. for example edge (u,v) is the edge
from u to v

\subsection{Weighted Graphs}
Weighted graphs are graphs in which it's edges contains a certain value attributed to certain
value such as cost, distance, quantity, etc... 

\subsection{Tree}
A tree is a an Undirected graph with no cycles. Equivalently it's a connected graph with 
n nodes and n-1 edges

\subsection{Rooted Trees}
A rooted tre is a tree with designated roote node where every edge either ponts away from
or towards the root node. When edges point away from the root node the graph is called 
arborescence or out tree and when the edges point towards the roote node the graph is called
anti-arborescence. 

\subsection{Directed Acyclic Graphs(DAGs)}
Directed Acyclic Graphs are Directed graphs with no cycles. These graphs are commonly used 
in representing structure with dependencies. Several efficient algorithms exist to operate
on DAGS



\end{document}
